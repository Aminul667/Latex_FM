\documentclass[11pt,a4paper]{article}
\usepackage[margin=1in]{geometry}
\usepackage{amsfonts,amsmath,amssymb,suetterl}
\usepackage{lmodern}
\usepackage[T1]{fontenc}
\usepackage{fancyhdr}
\usepackage{float}
\usepackage[utf8]{inputenc}

\usepackage{fontawesome}
\DeclareUnicodeCharacter{2212}{-}
\usepackage{mathrsfs}

\usepackage[nodisplayskipstretch]{setspace}

\setstretch{1.5}

\fancyfoot[C]{\thepage}

\renewcommand{\footrulewidth}{0pt}
\parindent 0ex
\setlength{\parskip}{1em}

\begin{document}
    \section*{MTH6141 Random Processes, 2013-14\\Solutions to Exercise Sheet 6}
    %
    \begin{enumerate}
        \item
        \begin{enumerate}
            \item  The "bare hands" approach is as follows. For any $k \in \mathbb{N,}$
            \begin{align*}
                \mathbb{P}(A+B = k)
                &= \sum_{i=0}^k \mathbb{P}(A = i)\mathbb{P}(B=k-i)\\
                &= \sum_{i=0}^k \frac{\lambda^i}{i!}e^{-\lambda}\, \frac{\mu^{k-i}}{(k-i)!}e^{-\mu}\\
                &= \frac{1}{k!}e^{-(\lambda + \mu)} \sum_{i=0}^k\binom{k}{i}\lambda^i\mu^{k-i}\\
                &= \frac{(\lambda + \mu)^k}{k!}\, e^{-(\lambda + \mu)}
            \end{align*}
            So $A+B \sim \text{Po}(\lambda + \mu)$. (An alternative, more elegant approach is via probability generating functions: see the Theorem 20 from \textit{Probability Models} and the following example.)
            %
            \item The cdf of the $\text{Exp}(\lambda)$ distribution is $F(t) = 1-e^{-\lambda t}$. The claim follows from the following chain of equalities:
            \begin{align*}
                \mathbb{P}(t > s+t\, | \, T>s)
                &= \frac{\mathbb{P}(T>s+t,T>s)}{\mathbb{P}(t>s)}\\
                &= \frac{\mathbb{P}(T>s+t)}{\mathbb{P}(t>s)}\\
                &= e^{-\lambda(s+t)}/e^{-\lambda s}\\
                &= e^{-\lambda t}\\
                &= \mathbb{P}(T>t).
            \end{align*}
            %
            \item Suppose $X(t) \sim \text{Po}(\lambda)$. Then
            \begingroup
            \allowdisplaybreaks
            \begin{align*}
                \mathbb{P}(Y(t) = k)
                &= \sum_{i=0}^\infty \mathbb{P}(X(t) = k+i)\mathbb{P}(Y(t) = k\, | \, X(t)=k+i)\\
                &= \sum_{i=0}^\infty \frac{\lambda^{k+i}}{(k+i)!}e^{-\lambda}\binom{k+i}{k}p^k(1-p)^i\\
                &= \sum_{i=0}^\infty \frac{\lambda^k \lambda^i}{k!i!}e^{-\lambda}p^k(1-p)^i\\
                &= \frac{(p\lambda)^k}{k!}e^{-\lambda} \sum_{i=0}^\infty \frac{((1-p)\lambda)^i}{i!}\\
                &= \frac{(p\lambda)^k}{k!}e^{-\lambda}e^{(1-p)\lambda}\\
                &= \frac{(p\lambda)^k}{k!}e^{-p\lambda}.\\
            \end{align*}
            \endgroup
            So $Y(t) \sim \text{Po}(p\lambda)$. (Alternatively, this question makes an attractive exercise in the use of probabilityt generating functions! See Theorem 21 from \textit{Probability Models}.)
        \end{enumerate}
        \item Denote the Poisson process (of bus arrivals) by $X(t)$.
        \begin{enumerate}
            \item Thus the waiting time is at most $5$ minutes is the same as the event $X(5) \geq 1$. So the probability that the waiting time is less than $5$ minutes is
            $$
            \mathbb{P}(X(5) \geq 1) = 1-\mathbb{P}(X(5)=0) = 1- e^{-5\lambda} = 1-e^{-2/3}
            $$
            %
            \item That the waiting time is between $5$ and $10$ minutes is event $X(5)=0\wedge X(10)\geq 1$, which is equivalent to $X(5) = 0\wedge X(10) - X(5) \geq 1$. Since the process is independent in disjoint intervals $(0,5]$ and $(5,10]$, the required probability is
            $$
            \mathbb{P}(X(5) = 0)\mathbb{P}(X(10)-X(5)\geq 1) = e^{-5\lambda}(1-e^{-5\lambda}) = e^{-2/3}(1-e^{-2/3}).
            $$
            \item Arguing as is (b), the conditional probability of a waiting time less than $20$ minutes is
            \begin{align*}
                \mathbb{P}(X(20) \geq 1\, | \, X(10) = 0)
                &= \mathbb{P}(X(20)-X(10)\geq 1\, | \, X(10) = 0)\\
                &= \mathbb{P}(X(20)-X(10)\geq 1)\\
                &= 1-\mathbb{P}(X(20)-X(10)=0)\\
                &= 1-e^{-10\lambda}\\
                &= 1-e^{-4/3}.
            \end{align*}
        \end{enumerate}
        %
        \item 
        \begin{enumerate}
            \item The key point here is that for any $s,t$ we have that $C(s + t) - C(s)\sim \text{Po}(t/5)$. Aside from that, we use several times the fact that the number of arrivals in the time interval $(10,20]$ and the number in the interval $(0,10]$ are independent r.v's.
            \begin{enumerate}
                \item $$\mathbb{P}(C(20) = 3)=\mathbb{P}(C(20)-C(0)=3)=\frac{4^3}{3!}e^{-4}=\frac{32}{3}e^{-4}.$$
                \item 
                \begingroup
                \allowdisplaybreaks
                \begin{align*}
                    \mathbb{P}(C(20)=3\, | \, C(10)=1)
                    &= \mathbb{P}(C(20)-C(10) = 2 \, | \, C(10)-C(0)=1)\\
                    &= \mathbb{P}(C(20)-C(10)=2)\\
                    &= \frac{2^2}{2!}e^{-2}\\
                    &= 2e^{-2}
                \end{align*}
                \endgroup
                \item 
                \begin{align*}
                    \mathbb{P}(C(10) = 1, C(20) = 3)
                    &= \mathbb{P}(C(10)-C(0)=1, C(20)-C(10) = 2)\\
                    &= \mathbb{P}(C(10)-C(0)=1)\mathbb{P}(C(20)-C(10)=2)\\
                    &= \frac{2^1}{1!}e^{-2}\times \frac{2^2}{2!}e^{-2}\\
                    &= 4e^{-4}
                \end{align*}
                \item 
                $$\mathbb{P}(C(20)=1\, | \, C(10)=3) = \mathbb{P}(C(20)-C(10)=-2)=0.$$
                \item 
                \begin{align*}
                    \mathbb{P}(C(10) 
                    &= 1\, | \, C(20) = 3) = \frac{\mathbb{P}C(10) = 1,\, C(20)=3}{\mathbb{P}(C(20)=3)}\\
                    &= \frac{4e^{-4}}{(32/3)e^{-4}}\\
                    &= \frac{3}{8}.
                \end{align*}
            \end{enumerate}
            \item The event that the second customer arrives in the first $15$ minutes is the same as the event that $C(15) \geq 2$. Now
            $$\mathbb{P}(C(15)\geq 2) = 1-e^{-3}-\frac{3^1}{1!}e^{-3}=1-4e^{-3}.$$
            \item If each customer spends $10$ minutes in the shop then the customers present at time $1$ hour are those who entered the shop in the interval $(50, 60]$. The number of these has $\text{Po}(10/5) = \text{Po}(2)$ distribution.
            \item In each minute, the arrivals are distributed $\text{Po}(1/5)$, independently of the arrivals in all other minutes, So, by I(a), the number of arrivals during the even minutes are distributed as $\text{Po}(30/5) = \text{Po}(6)$.
        \end{enumerate}
        %
        \item 
        \begin{enumerate}
            \item $P1$ is immediate.\\
            So to $P2$. For $s \geq 0,\ t > 0,$
            \begin{align*}
                \mathbb{Z}(s+t)-\mathbb{Z}(s)
                &= (X(s+t) + Y(s+t))-(X(s)+Y(s))\\
                &= (X(s+t)-X(s))+(Y(s+t)-Y(s)).\\
            \end{align*}
            But $X(s+t)-X(s) \sim \text{Po}(\lambda t)$ and $Y(s+t)-Y(s) \sim \text{Po}(\mu t)$, So by Question 1(a) from the previous exercise sheet, $\mathbb{Z}(s+t)-\mathbb{Z}(s)\sim \text{Po}((\lambda + \mu)t)$.\\
            For P3, since $X(t_2)-X(t_1),\ldots, X(t_n)-X(t_{n-1})$ and $Y(t_2)-Y(t_1),\ldots, Y(t_n)-Y(t_{n-1})$ are mutually independent random variables, so are
            $$
            (X(t_2)-X(t_1))+(Y(t_2)-Y(T_1)),\ldots,(X(t_n)-X(t_{n-1}))+(Y(t_n)-Y(T_{n-1}))
            $$
            Rearranging, we see that,
            $$
            Z(t_2)-Z(t_1), \ldots, Z(t_n)-Z(t_{n-1})
            $$
            are mutually independent.
            \item The arrivals of all trains is, by Part (b), a Poisson process of rate $6+14 = 20$ per hour, or $\lambda = \frac{1}{3}$ per minute. The waiting time (in minutes) is distributed $\text{Exp}(\lambda)$, so the expected waiting time is $\lambda^{-1} = 3$ minutes. The arrivals in the interval $(5, 10]$ are distributed $\text{Po}(5\lambda)=\text{Po}(\frac{5}{3})$, so the probability there is some arrival in that $5$-minute interval is $1-e^{-\frac{5}{3}}$.
        \end{enumerate}
        
    \end{enumerate}
\end{document}