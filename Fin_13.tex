\documentclass[11pt,a4paper]{report}
\usepackage[margin=1in]{geometry}
\usepackage{amsfonts,amsmath,amssymb,suetterl}
\usepackage{lmodern}
\usepackage[T1]{fontenc}
\usepackage{fancyhdr}
\usepackage{float}
\usepackage[utf8]{inputenc}
\usepackage{fontawesome}
\usepackage{enumerate}

\DeclareUnicodeCharacter{2212}{-}

\usepackage{mathrsfs}

\usepackage[nodisplayskipstretch]{setspace}
\setstretch{1.5}


\fancyfoot[C]{\thepage}

\renewcommand{\footrulewidth}{0pt}
\parindent 0ex
\setlength{\parskip}{1em}

\begin{document}
    \begin{center}
            \LARGE\textbf{MTH6121 Introduction to Mathematical Finance}\\
            Coursework 9
    \end{center}
    %
    Please hand in your solution of the \textbf{starred} exercises by \textbf{17.00 on Wednesday 7 December 2016} using the green Introduction to Mathematical Finance Collection Box on the second floor of the Mathematics Building. Don’t forget to put your \textbf{name} (with your \underline{surname} underlined), \textbf{student number} and your \textbf{tutorial group} on your solutions, and to \textbf{staple} them.\par 
    %
    \textbf{Exercise 1.} Suppose that you own a UK-based company having a liability of $\$ 1,000,000$ in half a year. In order to hedge the foreign exchange risk of your company (since you do not know the future exchange rate), you decide to go long a forward contract to buy $\$ 1,000,000$ for a (fixed) forward price $\pounds F$ in half a year. Suppose that the continuously compounded nominal interest rate in the UK is $r_{\pounds} = 1\%$ and in the US is $r_{\$} = 2\% $, while the exchange rate at time $0$ is $0.67\pounds / \$$. What is the fair value of $F$?\par 
    %
    \textbf{Exercise* 2.} At time $0$ investor A makes the following investment: she buys one share of company XYZ’s stock (worth $20$ pounds at this point in time) and, for $3$ pounds, one put option on the same stock with strike price $17$ pounds and expiration time $T = 1$ year.\par
    At the same time investor B also makes an investment: for $C$ pounds he buys a call option on company XYZ’s stock with the same strike price of $17$ pounds and the same expiration time $T = 1$ year, as investor A’s put option and also deposits $16.5$ pounds in a bank account where interest is compounded continuously with nominal rate $3\%$.
    %
    \begin{enumerate}[(a)]
        \item What is the payoff of investor A’s investment at time $T$ (explain in detail)?
        \item What is the payoff of investor B’s investment at time $T$ (explain in detail)?
        \item Use the Law of One Price together with your answers to (a) and (b) to deduce the price $C$ of the call option, under the assumption that no arbitrage opportunity exists.
    \end{enumerate}
    %
    \textbf{Exercise* 3.} Suppose that the following $3$ odds of $3$ possible outcomes of an experiment are available: $o_1 = 2/3,\ o_2 = 6/2$ and $o_3 = 10/1$. In other words, the return function is given by
    $$
    r_i(j)
    =
    \begin{cases}
        o_i & \text{if}\ j = i;\\
        -1 & \text{if}\ j \neq i.
    \end{cases}
    \quad
    \text{for $i, j = 1, 2, 3$.}
    $$
    \begin{enumerate}[(a)]
        \item Write down the gain function of the strategy $(x_1, x_2, x_3)$, i.e. when betting $x_i$ on outcome $i$.
        \item Show that there exist winning strategies $(x_1, x_2, x_3)$. The answer should be given in the form of specific inequalities for x1, x2 and x3, which yield winning betting strategies $(x_1, x_2, x_3)$ (Arbitrage opportunities).
        \item Is the strategy $(x_1, x_2, x_3) = (220, 100, 50)$ resulting in a risk free-profit? Explain.
        \item Is the strategy $(x_1, x_2, x_3) = (220, 100, 40)$ resulting in a risk free-profit? Explain.
    \end{enumerate}
    %
    \textbf{Exercise 4.} Let a return function be defined by $r_1(1) = 1,\ r_1(2) = 1,\ r_1(3) = −1,\ r_2(1) = −2,\ r_2(2) = −1,\ \text{and}\ r_2(3) = 4$. Show that the betting strategy $x = (3, 1)$ gives a risk free profit.    
\end{document}