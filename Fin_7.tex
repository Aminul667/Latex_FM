\documentclass[11pt,a4paper]{report}
\usepackage[margin=1in]{geometry}
\usepackage{amsfonts,amsmath,amssymb,suetterl}
\usepackage{lmodern}
\usepackage[T1]{fontenc}
\usepackage{fancyhdr}
\usepackage{float}
\usepackage[utf8]{inputenc}
\usepackage{fontawesome}
\usepackage{enumerate}

\DeclareUnicodeCharacter{2212}{-}

\usepackage{mathrsfs}

\usepackage[nodisplayskipstretch]{setspace}
\setstretch{1.5}


\fancyfoot[C]{\thepage}

\renewcommand{\footrulewidth}{0pt}
\parindent 0ex
\setlength{\parskip}{1em}

\begin{document}
    \begin{center}
        \LARGE\textbf{MTH6121 Introduction to Mathematical Finance}\\
        Coursework 6
    \end{center}
Please hand in your solution of the \textbf{starred} exercises by \textbf{17.00 on Wednesday 16 November 2016} using the green Introduction to Mathematical Finance Collection Box on the second floor of the Mathematics Building. Don’t forget to put your \textbf{name} (with your \underline{surname} underlined), \textbf{student number} and your \textbf{tutorial group} on your solutions, and to \textbf{staple} them.\par
\textbf{Exercise* 1.}
During your $3$-year studies at Queen Mary, you withdraw $1000$ pounds at the beginning of each month from your bank account, in order to pay for your living expenses and rent. You want to pay this back on a monthly basis after your graduation. You finish your studies after $3$ years. To avoid spoiling your newly gained financial independence after graduation, you want to pay back only $250$ pounds per month at the beginning of each month. Suppose that the nominal annual interest rate (compounded monthly) is $1.5\%$ throughout your studies and the repayment phase.
\begin{enumerate}[(a)]
    \item For how long after your graduation do you need to continue to make payments to clear your debt? How many payments will you have made?
    \item What will be the amount of your last repayment?
\end{enumerate}
%
\textbf{Exercise 2.} For this problem assume that the nominal annual interest rate (compounded monthly) is $1\%$. Suppose that you buy a laptop for 1000 pounds.
\begin{enumerate}[(a)]
    \item The store gives you a deal where you initially pay $250$ pounds and then make a payment $A$ at the end of each month for three years until your debt is paid off. What is the payment $A$?
    \item The store gives you a deal where you pay $20$ pounds (or the remaining balance if it is less than $20$ pounds) at the beginning of each month until your debt is paid off. How many payments will you make?
\end{enumerate}
%
\textbf{Exercise 3.} Let $P, r$ and $t$ be fixed positive constants. Prove that
$$\lim_{n\to \infty}P\left(1+\frac{r}{n}\right)^{nt}=Pe^{rt}$$
{[}Hint: Involve a logarithm on left-hand side and use the Taylor expansion of logarithm{]}

\end{document}