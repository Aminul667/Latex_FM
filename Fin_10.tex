\documentclass[11pt,a4paper]{article}
\usepackage[margin=1in]{geometry}
\usepackage{amsfonts,amsmath,amssymb,suetterl}
\usepackage{lmodern}
\usepackage[T1]{fontenc}
\usepackage{fancyhdr}
\usepackage{float}
\usepackage[utf8]{inputenc}
\usepackage{fontawesome}
\usepackage{enumerate}

\DeclareUnicodeCharacter{2212}{-}

\usepackage{mathrsfs}

\usepackage[nodisplayskipstretch]{setspace}
\setstretch{1.5}


\fancyfoot[C]{\thepage}

\renewcommand{\footrulewidth}{0pt}
\parindent 0ex
\setlength{\parskip}{1em}

\begin{document}
    \begin{center}
        \LARGE\textbf{MTH6121 Introduction to Mathematical Finance}\\
        Coursework 7 — Solutions
    \end{center}
    %
    \textbf{Exercise 1.} In order to take into account inflation, we use the inflation adjusted interest rate $r_a = 5.5\% − 4\% = 1.5\%$. Write
    $$ \beta = \frac{1}{1+r_a} = \frac{1}{1.015} = 0.985222. $$
    The present value of the salaries from job with company A is
    $$ 32000\beta + 33000\beta^2 + 36000\beta^3 = 97986.34 ,$$
    while the present value of the salaries from job with company B is
    $$6000 + 31000\beta + 31000\beta^2 + 31000\beta^3 = 96278.21 .$$
    The job with company A has a higher present value and is therefore a better deal.\par
    %
    \textbf{Exercise* 2.} 
    \begin{enumerate}[(a)]
        \item By definition, the internal rate of return of this investment is the unique solution $r$ of the equation
        $$
        1000 = \frac{0}{1+r} + \frac{500}{(1+r)^2} + \frac{0}{(1+r)^3} + \frac{750}{(1+r)^4}
        $$
        in the range $(−1,\infty)$. Using the substitution $x = (1 + r)^2$, the above equation yields
        $$
        1000 = \frac{500}{x} + \frac{750}{x^2},
        $$
        so
        $$
        x^2-\frac{1}{2}x-\frac{3}{4} = 0.
        $$
        Thus
        $$
        x = \frac{1}{4}(1 \pm \sqrt{13}),
        $$
        from which we choose only
        $$
        x = \frac{1}{4}(1 + \sqrt{13}),
        $$
        since $x > 0$, and so
        $$
        r = \sqrt{x} - 1 = \frac{1}{2}\sqrt{1+\sqrt{13}}-1 = 0.073.
        $$
        Thus the rate of return is $7.3\%$.
        %
        \item The capital gain of this investment is $500 + 750−1000 = 250$, and the corresponding amount of tax is $250 \cdot 0.18 = 45$. Thus the final payment will be 
        $750 − 45 = 705$. Thus, the internal rate of return of this investment in the presence of capital gains tax is the unique solution $r$ of the equation
        $$
        1000 = \frac{0}{(1+r)} + \frac{500}{(1+r)^2}+\frac{0}{(1+r)^3}+\frac{750}{(1+r)^4}
        $$
        in the range $(−1,\infty)$. Using the substitution $x = (1 + r)^2$, the above equation yields
        $$
        1000 = \frac{500}{x} + \frac{750}{x^2},
        $$
        so
        $$
        x^2 - \frac{1}{2}x - \frac{141}{200} = 0. 
        $$
        Thus
        $$
        x = \frac{1}{4} + \frac{1}{20}\sqrt{307},
        $$
        since $x > 0$, and so
        $$
        r = \sqrt{x}-1=\sqrt{\frac{1}{4}+\frac{1}{20}\sqrt{307}}-1 = 0.0611\ldots
        $$
        Thus the rate of return is $6.1\%$ in this case
    \end{enumerate}
    %
    \textbf{Exercise* 3.}
    \begin{enumerate}[(a)]
        \item The graph of $r(t)$ is piecewise linear. See Addition to Solution $7$ for the sketch of the graph for $a = 0.6$.
        \item The present value $P(0)$ of $P(3) = \pounds 2, 000$ to be received in $3$ years time is given by
        
        \begin{align*}
            P(0) 
            &= \text{exp}\, \left(-\int_0^3 r(u)\, du \right)P(3)\\
            &= \text{exp}\, \left(-\int_0^1 au\, du - \int_1^3 a\, du\right)2000\\
            &= \text{exp}\, \left(-\left[\frac{a}{2}u^2\right]_0^1-\left[au\right]_1^3\right)2000\\
            &= \text{exp}\, \left(-\frac{a}{2}-2a\right)2000\\
            &= \text{exp}\, \left(-\frac{5}{2}a\right)2000.
        \end{align*}

    \end{enumerate}
\end{document}