\documentclass[11pt,a4paper]{article}
\usepackage[margin=1in]{geometry}
\usepackage{amsfonts,amsmath,amssymb,suetterl}
\usepackage{lmodern}
\usepackage[T1]{fontenc}
\usepackage{fancyhdr}
\usepackage{float}
\usepackage[utf8]{inputenc}
\usepackage{fontawesome}
\usepackage{enumerate}

\DeclareUnicodeCharacter{2212}{-}

\usepackage{mathrsfs}

\usepackage[nodisplayskipstretch]{setspace}
\setstretch{1.5}


\fancyfoot[C]{\thepage}

\renewcommand{\footrulewidth}{0pt}
\parindent 0ex
\setlength{\parskip}{1em}

\begin{document}
    \begin{center}
        \LARGE\textbf{MTH6121 Introduction to Mathematical Finance}\\
        Coursework 10
    \end{center}
    %
    Please hand in your solution of the \textbf{starred} exercises by \textbf{17.00 on Wednesday 14 December 2016} using the green Introduction to Mathematical Finance Collection Box on the second floor of the Mathematics Building. Don’t forget to put your \textbf{name} (with your \underline{surname} underlined), \textbf{student number} and your \textbf{tutorial group} on your solutions, and to \textbf{staple} them.\par 
    %
    \textbf{Exercise 1.} Let a return function be defined by $r_(1) = 1, r_1(2) = 1, r_1(3) = −1, r_2(1) = −2, r_2(2) = −1,\ \text{and}\ r2(3) = 4$. Use the Arbitrage Theorem to show that there exists a betting strategy $x = (x_1, x_2)$ giving a guaranteed risk free profit.\par 
    %
    \textbf{Exercise* 2.} Suppose that the odds of $m$ possible outcomes of an experiment are $o_i > 0$, where $i = 1,\ldots , m$. In other words, the return function is given by
    $$
    r_i(j)
    =
    \begin{cases}
        o_i & \text{if $j=i$};\\
        -1 & \text{if $j \neq i$}.
    \end{cases}
    $$
    Use the Arbitrage Theorem to show that either
    $$
    \sum_{i=1}^m(1+o_i)^{-1}=1,
    $$
    or there is an arbitrage opportunity.\par 
    %
    \textbf{Exercise 3.} Assume that the interest rate is $r = 2\%$ per time period. Suppose that at time $0$, Vodafone stock price is traded at $\pounds 217$, and that at the next time period $1$, it is either traded at $\pounds 205$ or $\pounds 228$. We consider a European call option with maturity time period $1$ and strike price $\pounds 220$. What is the no-arbitrage price $C$ of the call option?\par 
    %
    \textbf{Exercise* 4.} Suppose that interest is compounded at nominal rate $r$ per time period. Consider a share with starting value $S(0) = S$ which can take on only two possible values at the next time period $t = 1$, either $S(1) = S_u$ or $S(1) = S_d$, for some values $S_d < S_u$.
    \begin{enumerate}[(a)]
        \item Explain why the relation
        \begin{equation}
            S_d \leq (1+r)S \leq S_u.
        \end{equation}
        is necessary in order for the market to allow for no-arbitrage opportunities. That is,show that there exists an arbitrage opportunity in the cases when this relation is violated, namely when either $(1 + r)S < S_d$ or $S_u < (1 + r)S$ holds.
        \item Derive a general formula for the price $C$ of a call option with strike price $K$ and expiration time $T = 1$ assuming that $S_d < K < S_u$.\\
        \underline{Hint:} \textsl{Use the Arbitrage Theorem}.
    \end{enumerate}
\end{document}