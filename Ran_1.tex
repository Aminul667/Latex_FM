\documentclass[11pt,a4paper,twoside]{report}
\usepackage[margin=1in]{geometry}
\usepackage{amsfonts,amsmath,amssymb,suetterl}
\usepackage{lmodern}
\usepackage[T1]{fontenc}
\usepackage{fancyhdr}
\usepackage{float}
\usepackage[utf8]{inputenc}
\usepackage{hyperref}
\usepackage{fontawesome}
\DeclareUnicodeCharacter{2212}{-}
\usepackage{mathrsfs}
\usepackage{array}
\usepackage[nodisplayskipstretch]{setspace}
\setstretch{1.5}


\fancyfoot[C]{\thepage}

\renewcommand{\footrulewidth}{0pt}
\parindent 0ex
\setlength{\parskip}{1em}

\begin{document}
\textbf{MTH6141 Random Processes, 2018, Exercise Sheet 2}\par
Please drop your answers in the Random Processes coursework box opposite Maths Reception by 5pm on Thursday 25th January.\par
You are strongly encouraged to attempt questions other than Q1 and Q4. Question 2 is a good “warm up” exercise, and Q3 provides experience in “modelling” a simple situation as a Markov chain.\par
Please send comments and corrections to \href{d.ellis@qmul.ac.uk}{d.ellis@qmul.ac.uk}.

\begin{enumerate}
%Q1
   \item Let $(X_0,\, X_1,\, X_2,\, \cdots)$ be the Markov chain with state-space $S=\{1,2\}$ and with transition matrix
   $$P = \begin{pmatrix}
   			\frac{2}{3} & \frac{1}{3}\\
   			\frac{1}{6} & \frac{5}{6}
   		\end{pmatrix}$$

   \begin{enumerate}
     \item By diagonalising $P$, find $P^5$
     \item Find $\mathbb{P}(X_5=1\, |\, X_0=1)$ and $\mathbb{P}(X_5 = 1\, |\, X_0 = 2)$.
     \item Find $\mathbb{P}(X_{20} = 1\, |\, X_0 = 1)$ and $\mathbb{P}(X_{20} = 1\, |\, X_0 = 2)$.  (Simplify your expressions enough to support your answer to part (d).)
	 \item Comment on your answers to (b) and (c).
   \end{enumerate}

%Q2
   \item A Markov chain with state-space ${1, 2, 3, 4}$ has transition matrix
   		$$\begin{pmatrix}
   			1 & 0 & 0 & 0\\
   			\frac{1}{4} & \frac{1}{4} & \frac{1}{4} & \frac{1}{4}\\
   			\frac{1}{6} & \frac{1}{6} & \frac{1}{6} & \frac{1}{2}\\
   			0 & 0 & 0 & 1
   		 \end{pmatrix}$$

   \begin{enumerate}
     \item Which states are absorbing?
     \item Find the probability that the chain ends up in state $1$, given that it starts in state $2$.
   \end{enumerate}
%Q3
   \item A standard die is rolled repeatedly until the sum of two consecutive rolls is exactly $4$.

   \begin{enumerate}
     \item Show how to model this process using a Markov chain with about $43$ states.
     \item Show how to model this process using a Markov chain with a substantially smaller number of states.
     \item Calculate the expectation of the number of rolls made in total.
   \end{enumerate}
   
%Q4
   \item  This is an exercise in applying ‘first-step analysis’. A Markov chain with state-space $\{1, 2, 3, 4, 5\}$ has transition matrix
   		$$\begin{pmatrix}
   			0 & \frac{2}{3} & \frac{1}{3} & 0 & 0\\
   			0 & \frac{1}{3} & \frac{1}{2} & 0 & \frac{1}{6}\\
   			0 & \frac{1}{2} & 0 & \frac{1}{4} & \frac{1}{4}\\
   			0 & 0 & 0 & 1 & 0 \\
   			0 & 0 & 0 & 0 & 1
   		 \end{pmatrix}$$
The process starts in state $1$.

   \begin{enumerate}
     \item State $5$ is an absorbing state. Which other state is absorbing?
     \item Calculate the probability that the process is absorbed in the state $5$.
     \item Calculate the expectation of the time of absorption.
     \item Calculate the expectation of the number of visits to state $2$ before absorption.
     \item Suppose that you gain $\pounds 10$ for each visit to state $2$ and lose $\pounds 5$ for each visit to state $3$. Calculate the expectation of the amount of money you gain.

   \end{enumerate}

%Q5
   \item Find a simple example of a Markov chain with a finite state space and with two absorbing states, for which the probability that the process eventually reaches an absorbing state is strictly between $0$ and $1$. Do the ‘first-step analysis’ equations for finding the probability of absorption in a particular absorbing state in your chain have a unique solution? If not, how would you identify the correct solution in your example?
   
\end{enumerate}
 
 
 
\end{document}