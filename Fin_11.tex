\documentclass[11pt,a4paper]{report}
\usepackage[margin=1in]{geometry}
\usepackage{amsfonts,amsmath,amssymb,suetterl}
\usepackage{lmodern}
\usepackage[T1]{fontenc}
\usepackage{fancyhdr}
\usepackage{float}
\usepackage[utf8]{inputenc}
\usepackage{fontawesome}
\usepackage{enumerate}

\DeclareUnicodeCharacter{2212}{-}

\usepackage{mathrsfs}

\usepackage[nodisplayskipstretch]{setspace}
\setstretch{1.5}


\fancyfoot[C]{\thepage}

\renewcommand{\footrulewidth}{0pt}
\parindent 0ex
\setlength{\parskip}{1em}

\begin{document}
    \begin{center}
        \LARGE\textbf{MTH6121 Introduction to Mathematical Finance}\\
        Coursework 8
    \end{center}
    Please hand in your solution of the \textbf{starred} exercises by \textbf{17.00 on Wednesday 30 November 2016} using the green Introduction to Mathematical Finance Collection Box on the second floor of the Mathematics Building. Don’t forget to put your \textbf{name} (with your \underline{surname} underlined), \textbf{student number} and your \textbf{tutorial group} on your solutions, and to \textbf{staple} them.\par 
    %
    \textbf{Exercise 1.} Consider a time-dependent interest rate of the form $r(t)=r_0t^\alpha$, where $r_0$ is the interest rate at time $t = 0$ and $\alpha$ is a small positive constant.
    \begin{enumerate}[(a)]
        \item Determine the yield curve $\bar{r}(t)$ and the present value of one pound received at time $T$.
        \item Let $\alpha = 0.01$ and let $r_0 = 4\%$. Assume that interest is compounded continuously and that there is no inflation. You moved into your new flat today and had to pay your landlord a deposit of $1000$ pounds. Your contract states that the deposit will be paid back in full (though without interest) in two years’ time. What is the present value of that payment in two years’ time? What is the present value of the profit your landlord makes? What is its value in two years’ time?
    \end{enumerate}
    %
    \textbf{Exercise 2.} (Continuation of Exercise 3 from Coursework 7) Suppose that interest is compounded continuously at rate
    $$
    r(t)
    =
    \begin{cases}
        a\, t & \text{if}\ 0 \leq t \leq 1,\\
        a & \text{if}\ t>1.
    \end{cases}
    $$
    where $a$ is constant with $0 < a < 1$.
    \begin{enumerate}[(a)]
        \item Determine the yield curve $\bar{r}(t)$ corresponding to $r(t)$.
        \item Determine the long-time behaviour of the yield, that is, find $\lim_{t \to 0}\bar{r}(t)$.
        \item If you put $2000$ pounds in the bank today how much will you have in the bank at time $t = 4$?
    \end{enumerate}
    %
    \textbf{Exercise 3.} One financial instrument that classifies as debt is the zero-coupon bond (ZCB). If an investor buys today a ZCB with face value $F$ and maturity $T$, then the investor will receive the amount quoted as face value $F$ at the maturity time $T$.\par 
    Suppose that the continuously compounded interest rate offered in the market is $2\%$. What is the no-arbitrage price of a ZCB today if it has face value $\pounds 1,000$ and maturity in $2$ years? Justify your answer.\par 
    %
    \textbf{Exercise* 4.} You are asked to determine the no-arbitrage price $F$ for which a share will be bought at time $T > 0$ (the so called forward price), given that at time $t$ with $0 \leq t < T$ the share pays a (certain) dividend amount $D$ to the share holder. Suppose that the share price at time $0$ is $S$ and that the interest is compounded continuously at nominal rate $r$.
    \begin{enumerate}[(a)]
        \item Show that either $S \geq De^{-rt}$ holds or there exists an arbitrage opportunity.
        \item Consider the following two investments
        \begin{enumerate}[(i)]
            \item At time $0$ enter a long position in a forward contract on one share, with forward price $F$ at time $T$ and (at the same time $0$) invest (deposit) $Fe^{-rT}+De^{-rt}$ in a bank account.
            \item At time $0$ buy one share. At time $t$, invest (deposit) the dividend amount $D$ (received by the share) in a bank account.
        \end{enumerate}
        Use the Law of One Price to show that either
        $$
        F = Se^{rT}-De^{r(T-t)}
        $$
        or there exists an arbitrage opportunity.
    \end{enumerate}
\end{document}