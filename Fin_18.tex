\documentclass[11pt,a4paper]{article}
\usepackage[margin=1in]{geometry}
\usepackage{amsfonts,amsmath,amssymb,suetterl}
\usepackage{lmodern}
\usepackage[T1]{fontenc}
\usepackage{fancyhdr}
\usepackage{float}
\usepackage[utf8]{inputenc}
\usepackage{fontawesome}
\usepackage{enumerate}
\usepackage{nopageno}

\DeclareUnicodeCharacter{2212}{-}

\usepackage[nodisplayskipstretch]{setspace}
\pagestyle{plain}
\setstretch{1.5}


% \fancyfoot[C]{\thepage}

\renewcommand{\footrulewidth}{0pt}
\parindent 0ex
\setlength{\parskip}{1em}

\begin{document}
    \begin{center}
        \LARGE\textbf{MTH6121 Introduction to Mathematical Finance}\\
        Coursework 11 — Solutions
    \end{center}
    %
    \textbf{Exercise 1.} Note that the Call option pays off only if $Su^yd^{5-y}>K$. This happens if and only if $y = 5$. In order to see this, observe that if $y = 4$, then $Su^4d = K$, while if $y \leq 3$, then $Su^yd^{5-y}<K$.
    \hspace*{0.5cm} Therefore,
    \begin{align*}
        C
        &= \frac{1}{(1+r)^5}\sum_{y=0}^5(Su^yd^{5-y}-K)^+P(Y=y)\\
        &= \frac{1}{(1+r)^5}\left(0+0+0+0+0+(Su^5-Su^4d)P(Y=10)\right)\\
        &= \frac{1}{(1+r)^5}\, Su^4(u-d)\binom{5}{5}\, p^5(1-p)^0\\
        &= \frac{1}{(1+r)^5}\, Su^4(u-d)p^5.
    \end{align*}
    In particular, if $u = 1.2,\ d = 0.8,\ S = 100,\ r = 0.05$, then
    $$
    p = \frac{1+r-d}{u-d} = 6.197.
    $$
    and thus
    $$
    C = \frac{100(1.2)^4\, 0.4\, (0.625)^5}{(1.05)^5} = 6.197\, .
    $$
    The no-arbitrage price of this Call option is therefore 6.20.\par 
    %
    \textbf{Exercise 2.} The no-arbitrage price $C$ of the call option in the multiperiod binomial model is given by
    $$
    C = \frac{1}{(1+r)^4}\sum_{y=0}^4(S(0)u^yd^{4-y}-K)^+P(Y=y)\, ,
    $$
    where $r = 0.05,\ K = 150,\ S(0) = S = 200,\ u = 1.5,\ d = 0.5$, and
    $$
    P(Y=y) = \binom{4}{y}p^y(1-p)^{4-y} \quad \text{with}\ p = \frac{1+r-d}{u-d}= 0.55\, . 
    $$
    Note that
    \begin{itemize}
        \item if $y = 4$, then $Su^4d^{4-4}=1012.5>K$;
        \item if $y = 3$, then $Su^3d^{4−3} = 337.5 > K$;
        \item if $y = 2$, then $Su^2d^{4−2} = 112.5 < K$.
    \end{itemize}
    Thus
    $$
    C = \frac{1}{(1+r)^4}\left((Su^4-K)p^4+(Su^3d-K)4p^3(1-p)\right)= 111.13\, ,
    $$
    that is, the no-arbitrage price of the call option in this model is $\pounds 113.13$.\par 
    %
    \textbf{Exercise 3.} Let $S(t)$ denote the price of the Bancroft Stock at time $t$, where $t$ is measured in years. We are told that $S(t)$ is geometric Brownian motion with drift parameter $\mu = 0.15$, volatility parameter $\sigma = 0.21$ and starting parameter $S = 38$, that is
    $$
    S(t) = S\, \text{exp}(\mu t +\sigma W(t)),
    $$
    where $W(t)$ denotes the Wiener process. Let $K = 41$ denote the strike price and let $T = 1/4$ denote the expiration time of the call option.
    \begin{enumerate}[(a)]
        \item Let $r = 0.05$ denote the continuously compounded interest rate. The Black-Scholes price $C$ of the call is given by the Black-Scholes Formula
        $$
        C = S \Phi (\omega)-Ke^{-rT}\Phi(\omega - \sigma \sqrt{T}),
        $$
        where
        $$
        \omega = \frac{rT+\frac{\sigma^2T}{2}-\log \frac{K}{S}}{\sigma\sqrt{T}}.
        $$
        Now
        $$
        \omega = −0.552127\quad \text{and}\quad \omega − \sigma \sqrt{T} = −0.657127\, ,
        $$
        so
        \begin{align*}
            C
            &= 38\Phi(-0.55)-41e^{-0.05/4}\Phi(-0.66)\\
            &= 38(1-\Phi(0.55))-41e^{-0.05/4}(1-\Phi (0.66))\\
            &= 38(1-0.7088)-41e^{-0.05/4}(1-0.7454)\\
            &= 0.76\, .
        \end{align*}
        Thus, the Black-Scholes price of the call option is $\pounds 0.76$.
        \item The option will be exercised if it yields a positive payoff at time $T$, that is if $S(T) > K$. Thus, the desired probability is
        \begin{align*}
            P(S(T) > K)
            &= P(S\, \text{exp}(\mu T + \sigma W(T)) > K)\\
            &= P\left(\mu T + \sigma W(T)> \log\frac{K}{S}\right)\\
            &= P\left(\frac{W(T)}{\sqrt{T}}>\frac{\log\frac{K}{S}-\mu T}{\sigma \sqrt{T}}\right)\\
            &= P\left(\frac{W(T)}{\sqrt{T}}>0.37\right)\\
            &= 1-\Phi(0.37) = 1-0.6443 = 0.3557\, .
        \end{align*}
        Here we have used the fact tha
        $$
        W(T)\sim N(0,T) \quad \Rightarrow \quad \frac{W(t)}{\sqrt{T}}\sim N(0,1)\, .
        $$
        Thus the probability that the call option will be exercised is $36\%$.
    \end{enumerate}
    %
    \textbf{Exercise 4.} Let $r = 0.04,\ T = \frac{4}{12} = \frac{1}{3},\ \sigma = 0.24,\ K = 42,\ \text{and}\ S = 40$. We denote by $S(t)$ the risk-neutral geometric Brownian motion with volatility parameter $\sigma = 0.21$ and starting parameter $S = 38$, that is
    $$
    S(t) = S\, \text{exp}\left(\left(r-\frac{1}{2}\sigma^2\right)t+\sigma W(t)\right)\, ,
    $$
    where $W(t)$ denotes the Wiener process.
    \begin{enumerate}[(a)]
        \item According to the Black-Scholes Formula, the price $C$ of the call option is
        $$
        C = S\Phi(\omega)-Ke^{-rT}\Phi(\omega-\sigma\sqrt{T})\, ,
        $$
        where
        $$
        \omega = \frac{rT+\frac{\sigma^2T}{2}-\log\frac{K}{S}}{\sigma \sqrt{T}}\, .
        $$
        Now
        $$
        \omega = −0.1866\quad \text{and}\quad \omega − \sigma \sqrt{T} = −0.3252\, ,
        $$
        giving
        $$
        C = 1.62\, .
        $$
        Thus, you will need to pay $\pounds 1.62$ for the call option.
        \item The probability that you will exercise this option at time $T = 1/3$, that is in $4$ months’ time, is given by
        \begin{align*}
            P(S(T)>K)
            &= P(S\, \text{exp}\left(\left(r-\frac{1}{2}\sigma^2\right)T+\sigma W(T)\right)>K)\\
            &= P\left(\left(r-\frac{1}{2}\sigma^2\right)T+\sigma W(T)> \log\frac{K}{S}\right)\\
            &= P\left(\frac{W(T)}{\sqrt{T}} > \frac{\log\frac{K}{S}-\left(r-\frac{1}{2}
            \sigma^2\right)T}{\sigma \sqrt{T}}\right)\\
            &= 1-\Phi(0.3252) = 0.373\, .
        \end{align*}
        Thus the probability that the call option will be exercised is $37\%$.
        \item Since the put and call options have the same underlying share, strike price and maturity, we can use the put-call parity to find the put option price $P$ through the formula
        $$
        P = C+Ke^{-rT}-S = 1.62+42\, e^{-0.04/3}-40 = 3.064
        $$
        Thus, you will need to pay $\pounds 3.06$ for the put option.
        \item The probability that you will exercise the put option at time $T = 1/3$, that is in $4$ months’ time, is given by
        $$
        P(S(T) < K) = 1 − P(S(T) \geq K) = 1 − 0.373 = 0.627\, .
        $$
        Thus the probability that the put option will be exercised is $63\%$
    \end{enumerate}
\end{document}