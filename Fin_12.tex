\documentclass[11pt,a4paper]{report}
\usepackage[margin=1in]{geometry}
\usepackage{amsfonts,amsmath,amssymb,suetterl}
\usepackage{lmodern}
\usepackage[T1]{fontenc}
\usepackage{fancyhdr}
\usepackage{float}
\usepackage[utf8]{inputenc}
\usepackage{fontawesome}
\usepackage{enumerate}

\DeclareUnicodeCharacter{2212}{-}

\usepackage{mathrsfs}

\usepackage[nodisplayskipstretch]{setspace}
\setstretch{1.5}


\fancyfoot[C]{\thepage}

\renewcommand{\footrulewidth}{0pt}
\parindent 0ex
\setlength{\parskip}{1em}

\begin{document}
    \begin{center}
        \LARGE\textbf{MTH6121 Introduction to Mathematical Finance}\\
        Coursework 8 - Solutions
    \end{center}
    %
    \textbf{Exercise 1.}
    \begin{enumerate}[(a)]
        \item The yield curve is given by
        $$
        \bar{r}(t)
        =
        \frac{1}{t}\int_0^t r(u)\, du 
        =
        \frac{1}{t}\int_0^t r_0u^\alpha\, du 
        =
        \frac{1}{t}\left[\frac{r_0}{\alpha + 1}u^{\alpha + 1}\right]_{u=0}^{u=\alpha}
        =
        \frac{r_0}{\alpha+1}t^\alpha,
        $$
        and the present value function is given by
        $$
        B(T)
        =
        \text{exp}(-T\bar{r}(T))
        =
        \text{exp}\left(-\frac{r_0}{\alpha+1}T^{\alpha+1}\right).
        $$
        \item Suppose that $\alpha = 0.01$ and $r_0 = 4\%$. The present value of your landlord’s repayment is
        $$
        1000\,\text{exp}(-2\bar{r}(2))
        =
        1000\,\text{exp}\left(-\frac{r_0}{\alpha+1}2^{\alpha+1}\right)
        =
        923.34
        $$
        pounds. Thus the present value of your landlords profit is $1000 − 923.34 = 76.66$ pounds and the value of your landlord’s profit in two years’ time is
        $$
        76.66\,\text{exp}\, (2\bar{r}(2))
        =
        76.66\,\text{exp}\left(\frac{r_0}{\alpha+1}2^{\alpha+1}\right)
        =
        3.02
        $$
        pounds.
    \end{enumerate}
    %
    \textbf{Exercise 2.}
    \begin{enumerate}[(a)]
        \item We need to distinguish two cases. For $t$ with $0 \leq t \leq 1$ we have
        $$
        \bar{r}(t)
        = \frac{1}{t}\int_0^t au\, du
        = \frac{1}{t}\left[\frac{a}{2}u^2\right]_0^t
        = \frac{1}{t}\frac{a}{2}t^2
        = \frac{a}{2}t\ ,
        $$
        while for $t$ with $t > 1$ we have
        $$
        \bar{r}(t)
        = \frac{1}{t}\left(\int_0^1 au\, du+\int_1^t a\, du\right)
        = \frac{1}{t}\left(\left[\frac{a}{2}u^2\right]_0^1+\left[au\right]_1^t\right)
        = \frac{1}{t}\left(\frac{a}{2}+at-a\right)
        = a\left(1- \frac{1}{2t}\right)\, .
        $$
        Thus
        $$
        \bar{r}(t)
        =
        \begin{cases}
            \frac{a}{2}t & \text{if}\ 0 \leq t \leq 1,\\
            a\left(1-\frac{1}{2t}\right) & \text{if}\ t>1.
        \end{cases}
        $$
        \item  We have
        $$
        \lim_{t \to \infty}\bar{r}(t)
        = \lim_{t \to \infty}a\left(1 - \frac{1}{2t}\right)
        = a.
        $$
        \item If you put $2000$ pounds in the bank today, at time $t = 4$ you will have
        $$
        P(4)
        = 2000\, \text{exp}(4\bar{r}(4))
        = 2000\, \text{exp}\left(4a\left(1-\frac{1}{8}\right)\right)
        = 2000\, \text{exp}\left(\frac{7}{2}a\right)
        $$
        pounds in the bank.
    \end{enumerate}
    \textbf{Exercise 3.} Suppose that the ZCB costs $x$ at time $t = 0$ and assume that the time $t$ is measured in years. We consider two strategies:\par 
    In the first strategy we buy the ZCB at $t = 0$ (we pay $x$) and therefore receive $\pounds 1, 000$ at $t = 2$ when the ZCB matures.\par 
    In the second strategy, we invest the amount
    $$
    \pounds e^{-0.02 \cdot 2 } 1,000
    = \pounds 960.79
    $$
    in the bank for $2$ years and therefore have $\pounds 1, 000$ at $t = 2$ in our bank account.\par 
    The two strategies have equal payoffs at time $t = 2$, therefore the present value of the payoffs of both strategies is the same. Therefore, by using the Law of One Price, the two strategies must cost the same at time $t = 0$. In conclusion, we have $x = \pounds 960.79$.\par 
    \textbf{Exercise* 4.}
    \begin{enumerate}[(a)]
        \item If $S < De^{-rt}$ an arbitrage opportunity arises as follows. Buy the share at time $0$, which you finance by borrowing $S$ from a bank. Therefore, you start with a zero cost or profit. At time $t$, your debt to the bank is $Se^{rt}$, , but the share pays dividends $D$, so you make a profit of $D-Se^{rt}$, which is positive by assumption.
        \item We first observe that the \underline{present value of the payoffs} of both investments is the same. In order to see this, note that the \textbf{first investment} results in having $F+De^{r(T-t)}$ in the bank account at time $T$. Now using the forward contract, you buy the share, at time $T$, by paying $F$ and you own the share. In total, your payoff is $De^{r(T-t)}+S(T)$. The \textbf{second investment} also results in owning the share (since you bought it at time 0) and having $De^{r(T-t)}$ in the bank account at time $T$ (from the investment of dividends at time $t$). In total, your payoff is $De^{r(T-t)}+S(T)$.\par 
        Next we observe that the \underline{cost of the \textbf{first investment}} is $Fe^{-rT}+De^{-rt}$,\\
        while the \underline{cost of the \textbf{second investment}} is $S$.\\
        Thus, by the Law of One Price (since the present value of the payoffs of both investments is the same), either
        $$
        Fe^{-rT}+De^{-rt}=S
        $$
        or there exists an arbitrage opportunity. In other words, either
        $$
        F = Se^{rT}+De^{r(T-t)}
        $$
        or there exists an arbitrage opportunity, as required.\\
        Note that $F$ is non-negative $(F \geq 0)$ due to (a), where it was proved that $S \geq D e^{−rt}$ holds under no-arbitrage.
    \end{enumerate}

    
\end{document}