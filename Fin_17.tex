\documentclass[11pt,a4paper]{article}
\usepackage[margin=1in]{geometry}
\usepackage{amsfonts,amsmath,amssymb,suetterl}
\usepackage{lmodern}
\usepackage[T1]{fontenc}
\usepackage{fancyhdr}
\usepackage{float}
\usepackage[utf8]{inputenc}
\usepackage{fontawesome}
\usepackage{enumerate}

\DeclareUnicodeCharacter{2212}{-}

\usepackage{mathrsfs}

\usepackage[nodisplayskipstretch]{setspace}
\setstretch{1.5}


\fancyfoot[C]{\thepage}

\renewcommand{\footrulewidth}{0pt}
\parindent 0ex
\setlength{\parskip}{0.5em}

\begin{document}
    \begin{center}
        \LARGE\textbf{MTH6121 Introduction to Mathematical Finance}\\
        Coursework 11
    \end{center}
    %
    You are not required to hand in solutions to exercises of Coursework 11.\par 
    \textbf{Exercise 1.} Consider a call option on a share following the multiperiod binomial model with $n = 5$ periods. The current share price is $S = 100$ (at time 0), the strike price of the option is $K = Su^4d$ and its maturity time $T = 5$. Calculate $C$ when $u = 1.2,\ d = 0.8$ and the nominal interest rate is $r = 5\%$ per period.\par 
    %
    \textbf{Exercise 2.} Consider the no-arbitrage price $C$ of a call option in the multiperiod binomial model with strike price $\pounds 150$ and maturity time equal to $4$ time periods. Find $C$, if the initial share price is $\pounds 200$, the interest rate is $5\%$ per period, and the remaining model parameters are $u = 1.5$ and $d = 0.5$.\\
    \hspace*{0.5cm}\underline{Hint:} \textsl{Use the fact that the risk-neutral probability of an upward movement is given by}
    $$
    p = \frac{1+r-d}{u-d}
    $$
    %
    \textbf{Exercise 3.} The price of the $F$. Bancroft \& Sons stock follows a geometric Brownian motion with parameters $\mu = 0.15$ and $\sigma = 0.21$. Presently, the stock’s price is $\pounds 38$. Consider a call option having three months until its maturity time and having a strike price of $\pounds 41$.
    \begin{enumerate}[(a)]
        \item If the continuously compounded interest rate is $5\%$, what is the Black-Scholes price of thecall?
        \item What is the probability that the call option will be exercised?
    \end{enumerate}
    %
    \textbf{Exercise 4.} The share price $S(t)$ of your favourite company follows the risk-neutral geometric Brownian motion with volatility parameter $\sigma = 0.24$. The continuously compounded interest rate is $4\%$. Today one share is worth $\pounds 40$. You decide to test the things you learnt in MTH6121 and buy a call option with a strike price $\pounds 42$ and maturity time of $4$ months.
    \begin{enumerate}[(a)]
        \item What will you have to pay for this option?
        \item What is the probability that you will exercise your call option in $4$ months’ time?
        \item Suppose that instead of a call option you wanted to buy a put option on the same underlying share, maturity time and strike price. What will you have to pay for this put option?\\
        \underline{Hine:} \textsl{Use the put-call parity.}
        \item What is the probability that you will exercise your put option in $4$ months’ time?
    \end{enumerate}
\end{document}