\documentclass[11pt,a4paper]{report}
\usepackage[margin=1in]{geometry}
\usepackage{amsfonts,amsmath,amssymb,suetterl}
\usepackage{lmodern}
\usepackage[T1]{fontenc}
\usepackage{fancyhdr}
\usepackage{float}
\usepackage[utf8]{inputenc}
\usepackage{fontawesome}
\usepackage{enumerate}

\DeclareUnicodeCharacter{2212}{-}

\usepackage{mathrsfs}

\usepackage[nodisplayskipstretch]{setspace}
\setstretch{1.5}


\fancyfoot[C]{\thepage}

\renewcommand{\footrulewidth}{0pt}
\parindent 0ex
\setlength{\parskip}{1em}

\begin{document}
    \begin{center}
        \LARGE\textbf{MTH6121 Introduction to Mathematical Finance}\\
        Coursework 6 — Solutions
    \end{center}
    \textbf{Exercise* 1.} Let $W = 1000$ denote your initial monthly withdrawals and let $A = 250$ denote your monthly repayments, both measured in pounds.
    %
    \begin{enumerate}[(a)]
        \item Let $t$ denote the time after which you will have repaid your debt measured in months after your graduation and let
        $$\beta = \frac{1}{1+\frac{0.015}{12}}$$
        Noting that there are $36$ months in $3$ years, the present value $PV_W$ of your withdrawals is
        $$PV_W = W + W\beta + W\beta^2+\ldots+W\beta^{36} = W\frac{1-\beta^{36}}{1-\beta},$$
        while the present value $PV_A$ of your repayments is
        $$PV_A = A\beta^{36}+A\beta^{37}+\ldots+A\beta^{36+t} = A\beta^{36}(1+\beta+\ldots+\beta^t)=A\beta^{36}\frac{1-\beta^{t+1}}{1-\beta}$$
        You will finally repay your debt when $PV_W = PV_A$, so
        $$W\frac{1-\beta^{36}}{1-\beta} = A\beta^{36}\frac{1-\beta^{t+1}}{1-\beta}.$$
        Solving for $t$ we find
        $$\beta^{t+1} = 1 - \frac{W}{A}(\beta^{-36}-1),$$
        So
        $$(t+1)\log\beta = \log\left(1-\frac{W}{A}(\beta^{-36}-1)\right),$$
        and thus
        $$t = \frac{\log\left(1-\frac{W}{A}(\beta^{-36}-1)\right)}{\log\beta}-1=161.77$$
        Thus you need to make repayments for $162$ months after your graduation, and you need to make $163$ payments in total.
        \item Let $R$ be the amount of your last payment in pounds. The present value $PV_R$ of your last payment is the present value of your withdrawals less the present value of your $162$ payments over $250$ pounds, that is,
        $$PV_R = (W+W\beta+\ldots+W\beta^{35}) - (A\beta^{36}+A\beta^{37}+\ldots +A\beta^{36+161})=W\frac{1-\beta^{36}}{1-\beta}-A\beta^{36}\frac{1-\beta^{162}}{1-\beta}.$$
        As your last payment will be made 36 + 162 = 198 months from now, its present value is
        $$PV_R=R\beta^{168}$$
        Thus
        $$R=PV_R\beta^{-198} = \left(W\frac{1-\beta^{36}}{1-\beta}-A\beta^{36}\frac{1-\beta^{162}}{1-\beta}\right) = 192.1195,$$
        that is, your last payment will be $192.12$ pounds.
    \end{enumerate}
    %
    \textbf{Exercise 2.} Write
    $$\beta = \frac{1}{1+\frac{0.01}{12}}$$
    \begin{enumerate}[(a)]
        \item Let $A$ be the amount of your monthly payments in pounds. Noting that the present value of your repayments must equal the 1000 pounds you owe, we must have
        $$1000 = 250 +A\beta+A\beta^2+\ldots + A\beta^{36}$$
        Thus
        $$750 = A\beta(1+\beta+\ldots +\beta^{35}) = A\beta\frac{1-\beta^{36}}{1-\beta},$$
        so
        $$A = \frac{750}{\beta}\frac{1-\beta}{1-\beta^{36}} = 21.156,$$
        that is, your monthly payments will be 21.16 pounds.
        \item Let $n$ be the number of payments you need to make. Then
        $$1000 = 20 + 20\beta + 20\beta^2+\ldots + 20\beta^(n-1),$$
        so 
        $$1000=20\frac{1-\beta^n}{1-\beta}\quad \Rightarrow \quad \beta^n = 50\beta-49.$$
        Solving for $n$ yields
        $$n = \frac{\log(50\beta - 49)}{\log \beta}=51.049.$$
        Thus, you need to make $52$ payments.
    \end{enumerate}
    \textbf{Exercise 3.} Recall that for $|x| < 1$,
    $$\log(1+x) = x - \frac{1}{2}x^2+\frac{1}{3}x^3-\ldots$$
    which is just the Taylor series of $\log(1 + x)$ around the point $0$. The above expansion together with the continuity of the exponential function now yields
    \begin{align*}
        \lim_{n \to \infty}P\left(1+\frac{r}{n}\right)^{nt} & = \lim_{n\to \infty}P\, \exp\left(\log\left(1+\frac{r}{n}\right)^{nt}\right)\\
        &= \lim_{n\to \infty}P\, \exp\left(nt\,\log\left(1+\frac{r}{n}\right)\right)\\
        &= \lim_{n\to \infty}P\, \exp\left(nt\, \left(\frac{r}{n}-\frac{1}{2}\left(\frac{r}{n}\right)^2+\frac{1}{3}\left(\frac{r}{n}\right)^3-\ldots\right)\right)\\
        &=\lim_{n\to \infty}P\, \exp\left(rt-\frac{1}{2}\frac{tr^2}{n}+\frac{1}{3}\frac{tr^3}{n^2}-\ldots\right)\\
        &= P\, \exp(rt-0+0-\ldots)\\
        &= P\, \exp(rt).
    \end{align*}
\end{document}